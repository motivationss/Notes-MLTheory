
\part{Generalization Theory}
\chapter{Supervised Learning Framework}

\section{Basic Setups}
In a supervised learning problem, we have a goal to predict a label given an input. Let \(S\)
denote the dataset \(\{(x_i, y_i)\}_{i=1}^n\) for 
\begin{itemize}
    \item \(x_i \in \mathcal{X}\), the inputs in the input space. 
    \item \(y_i \in \mathcal{Y}\), the label associated with \(x_i\) in the label space. 
\end{itemize}

We assume that the data are drawn \textbf{i.i.d.} from an unknown probability distribution 
\(\mathcal{P}\) over \(\mathcal{X} \times \mathcal{Y}\). We aim at learning a function mapping 
\(h \colon \mathcal{X} \to \mathcal{Y}\) (aka \emph{hypothesis}, \emph{predictor}, \emph{model}). 


To evaluate the performance of \(h\), we specify a loss function. A loss function 
\(\ell \colon \mathcal{Y}, \mathcal{Y} \to \mathbb{R}\) measures the difference between the predicted label 
and the groundtruth label. 

\begin{definition}[population risk]\label{def:population_risk}
    The \emph{population risk} of a \emph{hypothesis} \(h\) is its expected loss over the data 
    distribution \(\mathcal{P} \colon L_{\mathcal{P}} (h) = \mathbb{E}_{(x,y)\sim \mathcal{P}}[\ell (h(x), y)]\)
\end{definition}

\begin{eg}
    Examples of Loss Functions 
    \begin{itemize}
        \item Classification: 0-1 loss \(\ell(\yhat, y) = \indi(\yhat \neq y)\). 
        \item Regression: squared loss \(\ell(\yhat, y) = (\yhat - y)^2\)
    \end{itemize}
\end{eg}

% \newcommand{\H}{\mathcal{H}}

It is often impossible to consider all possible function mappings from \(\X \to \Y\). We 
usually only consider a \emph{hypothesis class} \(\Hcal\).  

\begin{eg}
    Examples of \(\Hcal\). 
    \begin{itemize}
        \item Linear function class: \(\Hcal = \{h_\theta | h_\theta (x) = \theta^T x, \theta \in \R\}\) 
        \item General parametric function class: \(\Hcal = \{h_\theta | h_\theta (x) = f(x,\theta), \theta \in \R^p\}\)
    \end{itemize}
\end{eg}


\section{Empirical Risk Minimization}

\begin{definition}[Empirical Risk]\label{def:erm}
    The \emph{empirical risk} of a hypothesis \(h\) is its average loss over the dataset \(S\)
    \[
        L_{s}(h) = \frac{1}{n} \sum_{i=1}^n \ell (h(x_i), y_i)    
    \]
\end{definition}

Empirical risk minimization (ERM) is any algorithm that minimizes the empirical risk over the hypothesis 
class \(\Hcal\). We denote a hypothesis returned by ERM as \(\hat{h}_{ERM}\), \emph{i.e.}: 

\[
    \hat{h}_{ERM} \in \text{argmin}_{h\in \Hcal} L_S (h)    
\]

If we assume \(h\) is independent of \(S\), then \(\expec_S[L_S(h)] = L_P(h)\). But in reality 
\(h\) and \(S\) are not independent. 


\section{Questions} 

In the supervised learning part of this course, we are mainly interested in the following two 
fundamental problems: 

\begin{itemize}
    \item \textbf{Statistical}: What guarantee do we have about \(L_P(\hat{h}_{ERM})\)? 
    \item \textbf{Optimization}: When may ERM be achieved efficiently? 
\end{itemize}


\chapter{Concentration Inequality}


Concentration inequalities are a mathematical tool to study the relation between population 
and empirical quantities. Consider the following main question: for i.i.d. random variables 
\(X_1, \ldots, X_n\), how does \(\frac{1}{n}\sum_{i=1}^n X_i\) relate to 
\(\expec[\frac{1}{n}\sum_{i=1}^n X_i] = \mu\)? 


\section{Chebyshev's Inequality}

\begin{lemma}[Markov's Inequality]\label{def:Markov}
    Let \(X\) be a non-negative random variable, then for all \(t > 0\), 
    \[
        \Pr (X \geq t) \leq \frac{\expec(x)}{t}  
    \]

    \begin{proof}
        \[
        \expec(X) \geq \Pr(X < t) * 0 + \Pr(X > t) * t 
        \]
    \end{proof}
\end{lemma}



\begin{theorem}[Chebyshev's Inequality]
    Let \(X\) be a random variable with finite expected value \(\mu\) and finite non-zero 
    variance \(\sigma^2\). Then for any real number \(t > 0\), 
    \[
        \Pr \midBrac{ \lvert X - \expec(X)\rvert \geq t } \leq \frac{\Var(X)}{t}  
    \]

    \begin{proof}
        \begin{align*}
            \Pr \midBrac{X - \expec[X] \geq t} &= \Pr \midBrac{(X - \expec[X])^2 \geq t^2} \\
                                               &= \frac{\expec\midBrac{(X - \expec[X])^2}}{t^2} \\ 
                                               &= \frac{\Var[X]}{t^2}
        \end{align*}
    \end{proof}
\end{theorem}

\begin{remark}
    \[
    \Pr \midBrac{ \absolute{X - \expec[X]} \geq t} \leq \frac{\expec[\absolute{X - \expec[X]}^P]}{t^P}
    \]
\end{remark}

\begin{corollary}
    Let \(x_1, \ldots, x_n\) be i.i.d. random variables such that \(\expec\midBrac{x_i} = \mu\), 
    \(\Var[x_i] = \sigma^2\). Then: 
    \[
        \Pr \midBrac{ \absolute{\frac{1}{n}\sum_{i=1}^n x_i - \mu} \geq t} \leq \frac{\sigma^2}{nt^2}  
    \]
\end{corollary}


\section{Hoeffding Inequality}

\begin{lemma}
    If \(X \in \midBrac{0, 1}\) a.s. Then,  

    \[
        \expec \midBrac{e^{\lambda \left(X - \expec[X]\right)}} \leq e^{\frac{\lambda^2}{8}}  
    \]
    for all \(\lambda \in \R\). 
\end{lemma}

\begin{proof}
    Let \(Z = X - \expec[X]\), then \(\expec[Z] = 0\). \\ 
    Define \(\psi(\lambda) \coloneqq \log \expec \midBrac{e^{\lambda Z}}\). \\ 
    Using the Taylor expansion to get that \(\psi(\lambda) = \psi (0) + \lambda \psi' (0) + \frac{\lambda^2}{2}\psi''(\lambda')\)
    where \(\lambda'\) is between 0 and \(\lambda\). 

    Here the first term \(\psi(0) = \log 1 = 0\), and the second term \(\lambda \psi'(0) = \expec[Z] = 0\). 
    The only thing we need to is to compute the third term. The idea is to bound the third term by \(1/4\). 

    Then 
    \begin{align*}
         \psi'(\lambda) &= \frac{\expec\midBrac{e^{\lambda Z} Z}}{e^{\lambda Z}} \color{red} = \expec [Y] \\ 
         \psi''(\lambda) &= \frac{\expec\midBrac{e^{\lambda Z} Z^2}}{e^{\lambda Z}} 
         - \left(\frac{\expec\midBrac{e^{\lambda Z}Z}}{\expec [e^{\lambda Z}]}\right)^2 \\ 
                        & \color{red} = \expec[Y^2] - \left(\expec[Y]\right)^2 \\ 
                        & \color{red} = \Var [Y]
    \end{align*}
    Where we can think of \(Y\) as a reweighted version of \(Z\), and we have that 
    \[
      dP_Y (x) = \frac{e^{\lambda x}}{\expec \midBrac{e^{\lambda Z}}} dP_Z (x)  
    \]

    \textcolor{red}{not finished yet...}
\end{proof}

\begin{remark}
    We also call such random variables \textbf{subgaussian} random variables. Another interpretation
    is that bounded random variables are subgaussian. 

    Another reminder is that the expectation is in the form of \textbf{Moment Generating Function}, 
    where \(e^x = \sum_{k=0}^\infty \frac{x^k}{k!}\). 
\end{remark}

\begin{theorem}[Hoeffding Inequality]\label{def:hoeffding_ineq}
    Let \(X_1, \ldots, X_n\) be i.i.d. random variables such that for each \(i\), \(X_i \in [0,1]\)
    a.s. Then for all \(t > 0\): 
    \begin{align*}
        &\Pr \midBrac{ \frac{1}{n}\sum_{i=1}^n X_i - \expec \midBrac{\frac{1}{n}\sum_{i=1}^n X_i} \geq t} \leq e^{-2nt^2} \\
        &\Pr \midBrac{ \frac{1}{n}\sum_{i=1}^n X_i - \expec \midBrac{\frac{1}{n}\sum_{i=1}^n X_i} \leq -t} \leq e^{-2nt^2}  
    \end{align*}
\end{theorem}

\begin{proof}
    Let us use \(\Xbar\) to denote \(\frac{1}{n}\sum_{i=1}^n X_i\). Then we have that 
    \begin{align*}
        \Pr \midBrac{ \Xbar - \expec[\Xbar] \geq t} 
        &= \Pr \midBrac{e^{\lambda(\Xbar - \expec[\Xbar] )} \geq e^{\lambda t}} \tag*{\(\lambda > 0\)} \\ 
        &\leq \frac{\expec \midBrac{e^{\lambda(\Xbar - \expec[\Xbar] )}}}{e^{\lambda t}} \tag*{\hyperref[def:Markov]{Markov}}  \\
        &= e^{-\lambda t} \cdot \expec \midBrac{e^{ \frac{\lambda}{n} \sum_{i=1}^n \left(X_i - \expec [X_i]\right) }} \\ 
        &= e^{-\lambda t} \cdot \expec \midBrac{\prod_{i=1}^n e^{\frac{\lambda}{n} \left(X_i - \expec[X_i]\right)}} \\ 
        &= e^{-\lambda t} \prod_{i=1}^n \expec \midBrac{e^{\frac{\lambda}{n} \left(X_i - \expec[X_i]\right)}} \tag*{Independence} \\ 
        &\leq e^{-\lambda t} \left(e^{\frac{1}{8} (\frac{\lambda}{n})^n}\right) \\ 
        &= e^{-\lambda t + \frac{\lambda^2}{8n}}  \\ 
        &= e^{-2nt^2} \tag*{let \(\lambda = 4nt\)}
    \end{align*}
    By symmetry, we complete the proof. 
\end{proof}

\begin{remark}[Equivalent Definition of Hoeffding Inequality]

    
    Let \(X_1, \ldots, X_n \in [0, 1]\) a.s. and independent, 
    \begin{align*}
        \forall \delta \in (0, 1), \ & \text{w.p.} \geq 1 - \delta \colon \ \ \frac{1}{n} 
        \sum_{i=1}^n X_i - \expec \midBrac{\frac{1}{n}\sum_{i=1}^n X_i} \leq \sqrt{\frac{\log \frac{1}{\delta}}{2n}} \\ 
        &\text{w.p.} \geq 1 - \delta \colon \ \ \expec \midBrac{\frac{1}{n}\sum_{i=1}^n X_i} - \frac{1}{n} 
        \sum_{i=1}^n X_i \leq \sqrt{\frac{\log \frac{1}{\delta}}{2n}}
    \end{align*}
\end{remark}

\section{Bounded Difference Concentration Inequality}

We are concerned with diffeormorphism (\emph{i.e.} change in one coordinate) and formally 
\[
    f(X_1, \cdots, X_n) \mapsto \expec \midBrac{f(X_1, \cdots, X_n)}
\]


\begin{theorem}[Mcdiarmid's inequality] \label{def:mcd_ineq}
    Suppose \(X_1, \ldots, X_n\) are independent random variables taking values in a set \(A\). 
    Let \(f \colon A^n \to \R\) be a function that satisfies the \emph{bounded difference} condition:

    \[
      \exists c_1, \ldots, c_n > 0 \text{s.t.} \forall \ x_1, \ldots, x_n \in A, x_i' \in A \\ 
      \absolute{f(x_1, \ldots, x_n) - f(x_1, \ldots, x_{i-1}, x_i', x_{i+1}, \ldots, x_n)} \leq c_i  
    \]
    Then, for all \(t > 0\),
    \[
        \Pr \midBrac{f(X_1, \ldots, X_n) - \expec \midBrac{f(X_1, \ldots, X_n)} \geq t}
        \leq e^{- \frac{2t^2}{\sum_{i=1}^n c_i^2}}
    \]
\end{theorem}

\begin{remark}
    If \(f(X_1, \ldots, X_n) = \frac{1}{n}\sum_{i=1}^n x_i\) and \(A = [0, 1]\), then 
    \(c_i = \frac{1}{n}\) and the bound recovers the \hyperref[def:hoeffding_ineq]{Hoeffding inequality} 
    as \(e^{-2nt^2}\). 
\end{remark}